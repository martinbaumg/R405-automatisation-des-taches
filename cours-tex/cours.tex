\documentclass[12pt, a4paper]{article}
\usepackage[francais]{babel}
\usepackage{caption}
\usepackage{graphicx}
\usepackage[T1]{fontenc}
\usepackage{listings}
\usepackage{geometry}
\usepackage[colorlinks=true,linkcolor=black,anchorcolor=black,citecolor=black,filecolor=black,menucolor=black,runcolor=black,urlcolor=black]{hyperref}

% \usepackage{mathpazo} --> Police à utiliser lors de rapports plus sérieux

\usepackage{fancyhdr}
\pagestyle{fancy}
\lhead{}
\rhead{}
\chead{}
\rfoot{\thepage}
\lfoot{Martin Baumgaertner}
\cfoot{}

\renewcommand{\headrulewidth}{0.4pt}
\renewcommand{\footrulewidth}{0.4pt}

\begin{document}
\begin{titlepage}
	\newcommand{\HRule}{\rule{\linewidth}{0.5mm}} 
	\center 
	\textsc{\LARGE iut de colmar}\\[6.5cm] 
	\textsc{\Large R405}\\[0.5cm] 
	\textsc{\large Année 2022-23}\\[0.5cm]
	\HRule\\[0.75cm]
	{\huge\bfseries Automatisation des tâches}\\[0.4cm]
	\HRule\\[1.5cm]
	\textsc{\large martin baumgaertner}\\[6.5cm] 

	\vfill\vfill\vfill
	{\large\today} 
	\vfill
\end{titlepage}
\newpage
\tableofcontents
\newpage
\section{CM 1 - 6 mars 2023}
\subsection{Introduction}
\textbf{Système d'exploitation MS-DOS :}\\

    \begin{itemize}
        \item mono-tâches
        \item mono-utilisateur
        \item dédidé pour les plateformes x86
        \item langage de script : batch
        \item script .bat 
        \item émulation partielle via cmd.exe\\
    \end{itemize}

En parallèle ils ont développé Windows NT, ce qui a permis d'avoir plusieurs utilisateurs, 
en gros ça a permis de faire du multi-tâches, une interface avec la bibliothèque Win32.\\

\subsection{PowerShell}
\textbf{PowerShell :}\\

    \begin{itemize}
        \item Intérpréteur de commandes
        \item repose sur sur le framework .NET
        \item commandes PowerShell, Unix et MS-DOS
        \item Tout est objet sous PowerShell
        \item script .ps1\\
    \end{itemize}

\subsection{PowerShell ISE}
\textbf{PowerShell ISE :}\\

    \begin{itemize}
        \item PowerShell Integrated Scripting Environment
        \item éditeur de script
        \item intègre un terminal PowerShell
        \item intègre un débogueur de script
        \item intègre un explorateur de modules\\
    \end{itemize}




\end{document}