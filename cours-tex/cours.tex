\documentclass[12pt, a4paper]{article}
\usepackage[francais]{babel}
\usepackage{caption}
\usepackage{graphicx}
\usepackage[T1]{fontenc}
\usepackage{listings}
\usepackage{geometry}
\usepackage{amsmath}
\usepackage{listings}
\usepackage[colorlinks=true,linkcolor=black,anchorcolor=black,citecolor=black,filecolor=black,menucolor=black,runcolor=black,urlcolor=black]{hyperref}

% \usepackage{mathpazo} --> Police à utiliser lors de rapports plus sérieux

\usepackage{fancyhdr}
\pagestyle{fancy}
\lhead{}
\rhead{}
\chead{}
\rfoot{\thepage}
\lfoot{Martin Baumgaertner}
\cfoot{}

\renewcommand{\headrulewidth}{0.4pt}
\renewcommand{\footrulewidth}{0.4pt}

\begin{document}
\begin{titlepage}
	\newcommand{\HRule}{\rule{\linewidth}{0.5mm}} 
	\center 
	\textsc{\LARGE iut de colmar}\\[6.5cm] 
	\textsc{\Large R405}\\[0.5cm] 
	\textsc{\large Année 2022-23}\\[0.5cm]
	\HRule\\[0.75cm]
	{\huge\bfseries Automatisation des tâches}\\[0.4cm]
	\HRule\\[1.5cm]
	\textsc{\large martin baumgaertner}\\[6.5cm] 

	\vfill\vfill\vfill
	{\large\today} 
	\vfill
\end{titlepage}
\newpage
\tableofcontents
\newpage
\section{CM 1 - 6 mars 2023}
\subsection{Introduction}
\textbf{Système d'exploitation MS-DOS :}\\

    \begin{itemize}
        \item mono-tâches
        \item mono-utilisateur
        \item dédidé pour les plateformes x86
        \item langage de script : batch
        \item script .bat 
        \item émulation partielle via cmd.exe\\
    \end{itemize}

En parallèle ils ont développé Windows NT, ce qui a permis d'avoir plusieurs utilisateurs, 
en gros ça a permis de faire du multi-tâches, une interface avec la bibliothèque Win32.\\

\subsection{PowerShell}
\textbf{PowerShell :}\\

    \begin{itemize}
        \item Intérpréteur de commandes
        \item repose sur sur le framework .NET
        \item commandes PowerShell, Unix et MS-DOS
        \item Tout est objet sous PowerShell
        \item script .ps1\\
    \end{itemize}

\subsection{PowerShell ISE}
\textbf{PowerShell ISE :}\\

    \begin{itemize}
        \item PowerShell Integrated Scripting Environment
        \item éditeur de script
        \item intègre un terminal PowerShell
        \item intègre un débogueur de script
        \item intègre un explorateur de modules\\
    \end{itemize}

\newpage
\section{CM 2 - 15 mars 2023}



\newpage
\section{TD 1 - 10 mars 2023}
\subsection{Question 1}
Ce que les commandes retourent :\\
\begin{enumerate}
    \item \textbf{Get-ChildItem} = Donne les dossiers dans le répertoire où on est 
    \item \textbf{Get-Service} = Donne les services qui sont lancés
    \item \textbf{Get-Command} = Donne les commandes qui sont disponibles
    \item \textbf{Get-Process} = Donne les processus qui sont lancés
    \item \textbf{Get-Help} = Donne l'aide pour une commande
    \item \textbf{Get-Location} = Donne le chemin du répertoire où on est
    \item \textbf{New-Item} = Crée un fichier ou un dossier
    \item \textbf{Get-Member} = Donne les membres d'un objet
    \item \textbf{Add-Content} = Ajoute du contenu dans un fichier
    \item \textbf{Get-ItemProperty} = Donne les propriétés d'un fichier
    \item \textbf{Get-Item} = Donne les informations d'un fichier
    \item \textbf{Measure-Object} = Donne les informations d'un fichier
    \item \textbf{Sort-Object} = Trie les objets
\end{enumerate}

\subsection{Question 2}
Pour créer un fichier appelé \textbf{MonFichier.txt} il faut faire la commande
suivante : \texttt{New-Item -Path U:/ Documents/ -Name MonFichier.txt -ItemType File}\\

Ensuite, pour écrire quelques lignes dedans il faut faire la commande suivante :\\

\texttt{Add-Content -Path U:/ Documents/MonFichier.txt -Value "Ligne 1"}\\

\subsection{Exercice 3}
Pour créer un dossier il faut faire : 
\texttt{New-Item -Path U:/ Documents/ -Name MonDossier -ItemType Directory}\\

Je crée donc l'arborescence demandée. 

\newpage
\subsection{Exercice 4}
Pour déplacer mon fichier \textbf{MonFichier.txt} dans le dossier \textbf{D4} il faut faire :\\

\texttt{Move-Item -Path U:/Documents/MonFichier.txt -Destination U:/D1/D2/D4/}\\

\subsection{Exercice 5}
Après avoir éxecuté la commande \texttt{Get-Acl monfichier.txt | Fomat-List}, 
on obtient toutes les informations sur des détails sur les droits d'accès. Je 
peux donc en déduire que le propriétaire du fichier est \textbf{e2100947}, le groupe
propriétaire est \textbf{UHA/Utilisateurs du domaine}. Les droits d'accès sur 
ce fichier sont Allow et modify pour l'utilisateur, et full-control pour les admins. 

\subsection{Exercice 6}
Pour compter toutes les commandes possibles, il faut faire de la commande suivante :\\

\texttt{Get-Command | Measure-Object} : On obtient le résultat qui est 2492.\\

\subsection{Exercice 7}
Pour afficher les 10 premières lignes du fichier qu'on a crée avec la commande 
Select-Object, il faut faire la commande suivante :\\

\texttt{Get-Content -Path U:/Documents/D1/D2D/D4/MonFichier.txt | Select-Object -First 10}\\

\subsection{Exercice 8}
Pour afficher la troisème ligne du fichier qu'on a crée avec la commande Select-Object,
il faut faire la commande suivante :\\

\texttt{Get-Content -Path U:/Documents/D1/D2D/D4/MonFichier.txt | Select-Object -Skip 2 -First 1}\\

\subsection{Exercice 9}
Pour sélectionner dans ce fichier un pattern dont les erreurs seront et le résultat 
seront rédigés dans un fichier en utilisant la commande Select-String, il faut faire
la commande suivante :\\

\texttt{Get-Content -Path U:/Documents/D1/D2D/D4/MonFichier.txt | Select-String -Pattern "erreur" -ErrorAction SilentlyContinue | Out-File -FilePath U:/Documents/D1/D2D/D4/Resultat.txt}\\


\section{TD 2 - 13 mars 2023}
\subsection{Exercice 1}
Pour écrire l’expression régulière qui permet de vérifier qu’un mot est au moins composé de quatre lettres il 
faut faire la commande suivante :\\

\texttt{\^[a-zA-Z]{4,}\$}\\

par exemple, si on veut vérifier que le mot "test" est bien composé de de quattre lettres on écrit : 
\texttt{"test" -match "\^[a-zA-Z]{4,}\$"}\\

La commande renvoie \texttt{True} car le mot "test" est bien composé de quatre lettres.\\


\subsection{Exercice 2}
Pour écrire l’expression régulière qui permet de tester qu’il existe un espace entre "je test"
il faut faire la commande suivante :\\

\texttt{"je test" -match "je\textbackslash s+test"}\\

La commande renvoie \texttt{True} car il y a bien un espace entre "je" et "test".\\

\newpage
\subsection{Exercice 3}
Pour tester si un nom est composé je peux écrire le script suivant :\\

\begin{lstlisting}[language=Python]

    $nom = Jean-Claude

    if ($nom -match "^[a-zA-Z]+-[a-zA-Z]+$") {
        Write-Host "nom compose"
    } else {
        Write-Host "Le prenom n'est pas compose."
    }

\end{lstlisting}


\subsection{Exercice 4}
Pour tester si une plaque est valide aux normes françaises je peux écrire le script
suivant pour la plaque proposé :\\

\begin{lstlisting}[language=Python]
    $plaque = "AX-624-LP"

    if ($plaque -match "^[A-Z]{2}-[0-9]{3}-[A-Z]{2}$") {
        Write-Host "La plaque d'immatriculation est valide."
    } else {
        Write-Host "La plaque d'immatriculation est invalide."
    }
\end{lstlisting}

\subsection{Exercice 5}

Pour teser si une adresse mail de type \texttt{nom.prenom@uha.fr} je peux écrire le script suivant :

\begin{lstlisting}[language=Python]
    
    $mail = "martin.baumgaertner@uha.fr"

    if ($mail -match "^[a-zA-Z]+.[a-zA-Z]+@uha.fr$") {
        Write-Host "L'adresse mail est valide."
    } else {
        Write-Host "L'adresse mail est invalide."
    }

\end{lstlisting}

\newpage
\section{TD 3 - 14 mars 2023}

\subsection{Exercice 1}
Pour créer un script qui juge en fonction d'une note, je crée simplement 
ce code : 
\begin{lstlisting}
    echo "Entrez votre note : "
    read note

    if (( $note >= 16 && $note <= 20 ))
    then
        echo "Tres bien"
    elif (( $note >= 14 && $note < 16 ))
    then
        echo "Bien"
    elif (( $note >= 12 && $note < 14 ))
    then
        echo "Assez bien"
    elif (( $note >= 10 && $note < 12 ))
    then
        echo "Moyen"
    elif (( $note >= 8 && $note < 10 ))
    then
        echo "Insuffisant"
    else
        echo "Mediocre"
    fi
    
\end{lstlisting}

Pour les autres exercices, voit tout simplement les fichiers dans le répertoire 
TD1. 


\end{document}